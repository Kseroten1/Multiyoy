
\section{Wprowadzenie}
Autor pracy zdecydował się na stworzenie projektu, który bazuje na zasadach gry \textit{Slay}, ponieważ od zawsze miał zamiar stworzyć własną grę komputerową, a \textit{Slay} jest jedną z gier, która szczególnie przykuła jego uwagę. Jest to tytuł strategiczny, w którym kluczowe znaczenie mają przemyślane decyzje, rozwój zasobów oraz umiejętność przewidywania ruchów przeciwnika. Te elementy strategii i rozgrywki bardzo odpowiadają autorowi, jednak dostrzega on również pewne niedociągnięcia w \textit{Slay}, które sprawiają, że gra nie spełnia wszystkich jego oczekiwań.

Gra \textit{Slay} oferuje tryb wieloosobowy, ale wyłącznie w formie tzw. rozgrywki „hot-seat” – gracze na zmianę wykonują swoje ruchy na tym samym urządzeniu. Dla autora taki tryb wieloosobowy jest niewystarczający, ponieważ brak w nim możliwości rozgrywki w czasie rzeczywistym z innymi graczami online. Autor uważa, że tryb online pozwoliłby na głębszą interakcję między graczami i nadał grze zupełnie nowy wymiar. Jest przekonany, że gra strategiczna tego typu – rozgrywana w czasie rzeczywistym – nie tylko zapewniłaby większe wyzwania, ale też wzbogaciłaby doświadczenie i zaangażowanie graczy.

Wybór pracy opartej na grze \textit{Slay} stanowi także odpowiedź na chęć zdobycia wiedzy i doświadczenia w zakresie tworzenia gier, szczególnie w środowisku, które umożliwia implementację trybu wieloosobowego online. Autor chce połączyć swoją pasję do strategii z umiejętnościami programistycznymi, jednocześnie zaspokajając swoje ambicje, by stworzyć grę bardziej rozbudowaną i angażującą niż pierwowzór. Realizacja tego projektu to także doskonała okazja, by zagłębić się w tematykę programowania sieciowego, synchronizacji danych oraz projektowania przyjaznych dla gracza interfejsów.

Wprowadzenie trybu online wymaga rozważenia różnych aspektów interakcji między graczami, takich jak balans rozgrywki, przeciwdziałanie oszustwom, a także zapewnienie stabilności i płynności działania aplikacji – te wyzwania są dla autora fascynującym polem do rozwoju.

\newpage
\section{Cel pracy}

Ostatecznym celem autora jest stworzenie gry, która zaoferuje innym graczom nie tylko znane już mechanizmy, które są charakterystyczne dla \textit{Slay}, ale także nowe doświadczenia i emocje płynące z możliwości rozgrywki w trybie wieloosobowym online. Praca nad tym projektem jest więc dla niego zarówno realizacją wieloletniego marzenia, jak i cenną lekcją w zakresie programowania oraz projektowania gier.

Aby zrealizować ten cel, autor stawia przed sobą szereg mniejszych, szczegółowych zadań. Jednym z nich jest implementacja systemu generowania map, który umożliwi tworzenie różnorodnych i zbalansowanych plansz do gry. Kluczowym elementem jest także opracowanie systemu ekonomii, który nada głębię mechanikom gry, pozwalając graczom na strategiczne zarządzanie zasobami. Ponadto przewidziane jest wdrożenie mechanizmów zarządzania jednostkami, które będą odzwierciedlać zasady gry \textit{Slay}, ale również uwzględniać nowe możliwości taktyczne. Istotnym wyzwaniem będzie zaprojektowanie stabilnego trybu wieloosobowego online, zapewniającego płynną interakcję między graczami.

W procesie tworzenia gry zostaną również zaimplementowane elementy ułatwiające nawigację, takie jak intuicyjny interfejs użytkownika oraz algorytmy wspomagające balans rozgrywki. Wszystkie te cele mają na celu stworzenie gry, która będzie nie tylko funkcjonalna, ale także angażująca i satysfakcjonująca dla graczy.

\newpage
\section{Teza Pracy}

Dynamiczny rozwój technologii mobilnych oraz rosnące zapotrzebowanie na innowacyjne formy rozrywki sprawiają, że gry komputerowe oparte na mechanikach strategicznych zyskują coraz większą popularność. Niniejsza praca ma na celu wykazanie, że możliwe jest efektywne zaprojektowanie i zaimplementowanie gry komputerowej, inspirowanej mechanikami rozgrywki znanymi z gry \textit{Slay}, przy jednoczesnym zachowaniu jej strategicznego charakteru i intuicyjności. W ramach realizacji tego projektu zostaną przeanalizowane kluczowe aspekty tworzenia gry, takie jak projektowanie interfejsu użytkownika, optymalizacja mechanizmów rozgrywki oraz implementacja funkcji wspierających rozgrywkę wieloosobową online. Wynikiem pracy będzie gra, która nie tylko wiernie odwzoruje istotne cechy „Slay”, ale również wprowadzi elementy innowacyjne, zwiększając jej atrakcyjność w cyfrowym środowisku.
