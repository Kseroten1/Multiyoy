\documentclass[oneside,a4paper,12pt]{report} %,draft,openright

\usepackage[polish]{babel}
\usepackage{polski}
\usepackage{ragged2e}
\usepackage{blindtext}
\usepackage[utf8]{inputenc} 
\usepackage{gensymb}
\usepackage{graphicx}
\graphicspath{ {images/} }
\usepackage{epsf,graphicx}
%\usepackage{subfigure}
\usepackage{latexsym,amssymb}
\usepackage{setspace,cite}
\usepackage{indentfirst}
\usepackage{mathtools}
\usepackage[justification=centering]{caption}
\usepackage{multirow}
\usepackage[figuresright]{rotating}
\usepackage{enumitem}
\usepackage{xparse}
\usepackage{tabularx}
\usepackage[normalem]{ulem} 
\usepackage{titlesec}
\usepackage{caption}
\usepackage{subcaption}
\usepackage{amssymb} % for \longrightarrow
\usepackage[T1]{fontenc} % for \symbol{92} 
\usepackage{listings}
\usepackage[hidelinks]{hyperref}
\makeatletter
\newcommand{\unchapter}[1]{%
  \begingroup
  \let\@makechapterhead\@gobble % make \@makechapterhead do nothing
  \chapter{#1}
  \endgroup
}
\makeatother

\titleformat{\chapter}[display]
  {\normalfont\bfseries}{}{0pt}{\Huge}
\pagestyle{headings}

\NewDocumentEnvironment{sidebyside}{O{1.50} o +m +m}{%
  \noindent\begin{minipage}[t][][t]{#1\linewidth}%
  #3% Content of the first minipage
  \end{minipage}%
  \hfill%
  \noindent\begin{minipage}[t][][t]{\IfValueTF{#2}{#2}{#1}\linewidth}%
  #4% Content of the second minipage
  \end{minipage}\\% newline is important, it allows \hfill to work correctly, try removing it ;)
}

\usepackage[usenames, dvipsnames]{color}
% for margins left, right top bottom
\usepackage{anysize}
\marginsize{3cm}{3cm}{0.5cm}{1cm}
\let\origdoublepage\cleardoublepage    %%komenda wstawiająca czyste kartki
\newcommand{\clearemptydoublepage}{%
  \clearpage
  {\pagestyle{empty}\origdoublepage}%
}
\let\cleardoublepage\clearemptydoublepage

%\usepackage{draft} %draft option - doesn't put full figures in -
            % useful when editing

%does the headers on the pages - keep in
\usepackage{fancyhdr}
%omitting any of these makes the thesis compile without the omitted
%chapter - good for editing single chapters.
%\includeonly{header,appendix}
\usepackage{xcolor}
\usepackage{listings}

% Kolory dla kodu
\definecolor{codegray}{rgb}{0.9,0.9,0.9}
\definecolor{commentgreen}{rgb}{0.25, 0.5, 0.35}
\definecolor{keywordblue}{rgb}{0.1, 0.1, 0.9}
\definecolor{stringpurple}{rgb}{0.6, 0.0, 0.6}
\definecolor{numberorange}{rgb}{0.9, 0.4, 0.0}

% Główny styl dla kodów
\lstdefinestyle{mycodestyle}{
    backgroundcolor=\color{codegray},
    commentstyle=\color{commentgreen}\itshape,
    keywordstyle=\color{keywordblue}\bfseries,
    stringstyle=\color{stringpurple},
    numberstyle=\tiny\color{numberorange},
    basicstyle=\ttfamily\small,
    breaklines=true,
    frame=single,
    tabsize=2,
    showstringspaces=false,
    numbers=left,
    numbersep=5pt,
    xleftmargin=1em,
    framexleftmargin=1em,
}

% JavaScript i GLSL traktujemy jako podobny język (użyjemy js)
\lstdefinelanguage{GLSL}{
  keywords={in, out, uniform, layout, float, vec2, vec3, vec4, mat3, mat4, if, else, for, while, return, true, false, const},
  sensitive=true,
  comment=[l]{//},
  morecomment=[s]{/*}{*/},
  morestring=[b]",
}

% Aktywujemy nasz styl jako domyślny
\lstset{style=mycodestyle}
% Definicja języka JavaScript dla listings
\lstdefinelanguage{JavaScript}{
  keywords={const, var, let, function, if, else, for, while, return, class, new, try, catch, import, from, export, default, true, false, null, undefined},
  sensitive=true,
  morecomment=[l]{//},
  morecomment=[s]{/*}{*/},
  morestring=[b]",
  morestring=[b]',
  morestring=[b]`,
}
\lstset{literate=
  {ą}{{\k{a}}}1 {ć}{{\'c}}1 {ę}{{\k{e}}}1 {ł}{{\l{}}}1
  {ń}{{\'n}}1 {ó}{{\'o}}1 {ś}{{\'s}}1 {ż}{{\.z}}1 {ź}{{\'z}}1
  {Ą}{{\k{A}}}1 {Ć}{{\'C}}1 {Ę}{{\k{E}}}1 {Ł}{{\L{}}}1
  {Ń}{{\'N}}1 {Ó}{{\'O}}1 {Ś}{{\'S}}1 {Ż}{{\.Z}}1 {Ź}{{\'Z}}1
}
\begin{document}
\newpage
%Puts page numbering of preamble in roman and of main body of thesis in
%arabic. Also defines how chapters and sections are made
\pagenumbering{arabic}
\setcounter{page}{1} \pagestyle{fancy}
\renewcommand{\chaptermark}[1]{\markboth{\chaptername%
\ \thechapter:\,\ #1}{}}
\renewcommand{\sectionmark}[1]{\markright{\thesection\,\ #1}}
%DEFINES TITLE PAGE, and contains abstract, acknowledgements, etc.
%%%%%%%%%%%%%%%%%%%%%%%%%%%%%%%%%%%%%%%%%%%%%%%%%%%%%%%%%%%%%%%%%%%%%%%%%%%
% This is a sample header for a sample dissertation. Fill in the name,
% and the other information. LaTeX will work out the table of
% content, the list of figures and of tables for you.
%%%%%%%%%%%%%%%%%%%%%%%%%%%%%%%%%%%%%%%%%%%%%%%%%%%%%%%%%%%%%%%%%%%%%%%%%%%

\newpage
\thispagestyle{empty}

% ******* Title page *******
% **************************

\begin{onehalfspacing}
    \begin{center}

        \centering
        \includegraphics[keepaspectratio,scale=1.1]{./figures/logoAmw.jpg} \\[.5cm]


        {\fontsize{17}{17}\selectfont
        \textsc{\textbf{Akademia Marynarki Wojennej \\[.25cm]
        im. Bohaterów Westerplatte  \\[.25cm]
        Wydział Mechaniczno- Elektryczny } \\[.25cm]
        {Praca inżynierska\\[1.7cm]}} %%%%%%%%%%%%%%% TUTAJ ZMIEŃ 




        \large
        Projekt i implementacja aplikacji rozrywkowej na podstawie gry \textit{Slay} \\[2.3cm] %%%%%%%%%%%%%%% TUTAJ ZMIEŃ %%%%%%%%%%%%%%%
% Jeśli tytuł pracy zajmuje 2 linijki, wartość [2.3cm] zamieniamy na [3.1cm], jeśli tylko jedną - na [3.9cm] i odwrotnie - zwiększając liczbę linijek o jedną (do czterech) zmieniamy na [1.5cm] itd.

        \large
            \begin{flushleft}
                Wykonawca: Damian Klein  \\ %%%%%%%%%%%%%%% TUTAJ ZMIEŃ 
            \end{flushleft}


            \sidebyside[0.45][0.45]{Kierownik pracy }{Kierownik Katedry}
            \sidebyside[0.45][0.45]{Dr. Inż. Jan Masiejczyk}{\textcolor{green}{ Wpisać stopień, imię i nazwisko kierownika katedry}}
            \vspace{1cm}

            \begin{tabular}{c c c}
                \multicolumn{3}{l}{Konsultant: \textcolor{blue}{Wpisać stopień, imię i nazwisko konsultanta}} \\
                Ocena pracy dyplomowej   &  .......................... & .......................... \\
                \textcolor{white}{a} & \footnotesize słownie & \footnotesize data i podpis \\
                & & \footnotesize Przewodniczącego Komisji Egzaminacyjnej
            \end{tabular}

            \vspace{3cm}
        }
    \end{center}
\end{onehalfspacing}

\singlespacing
\newpage
%\thispagestyle{empty}
\sidebyside[0.4][0.4]{............................. }{Gdynia, dnia ........................ r.}

\sidebyside[0.4][0.4]{Imię i Nazwisko }{}

\sidebyside[0.4][0.4]{............................. }{}

\sidebyside[0.4][0.4]{Numer albumu}{}

\vspace{2cm}
\begin{center} \textbf{OŚWIADCZENIE} \end{center}

\justifying
Oświadczam, że przedłożoną do egzaminu pracę dyplomową pt. ............ kończącą studia I stopnia/ studia II stopnia/ Jednolite Studia Magisterskie*) napisałem(am) samodzielnie. Przy wykonywaniu pracy nie zlecałem(am) jej opracowania ani żadnej jej części innym osobom, jak też nie skopiowałem(am) cudzych opracowań  i przestrzegałem(am) postanowień Ustawy z dnia 4 lutego 1994 r. o prawie autorskim  i prawach pokrewnych (Dz. U. z 2018 r. poz. 1191 z późn. zmianami). \vspace{0.5cm}

Ponadto oświadczam, iż treści zaczerpnięte z literatury przedmiotu są oznaczone w tekście oraz w przypisach, w sposób ogólnie przyjęty dla prac naukowych. \vspace{0.5cm}

Jednocześnie przyjmuję do wiadomości, że gdyby powyższe oświadczenie okazało się nieprawdziwe, jestem świadomy(a) zasadności cofnięcia decyzji o wydaniu mi dyplomu. \vspace{0.5cm}

Wyrażam/ Nie wyrażam zgodę/ zgody na udostępnienie mojej pracy dyplomowej czytelnikom. \vspace{0.5cm}

Przekazuję moją pracę dyplomową do Ogólnego Repozytorium Prac Dyplomowych. \vspace{0.5cm}

\vspace{2cm}
\sidebyside[0.4][0.45]{ }{.......................................................}

\sidebyside[0.4][0.45]{}{Podpis osoby składającej oświadczenie}
\mbox{}

\doublespacing
\newpage
\thispagestyle{empty}
\mbox{}

%\pagestyle{empty}
\pagenumbering{Roman}
\setcounter{page}{0} \pagestyle{plain}


\tableofcontents

\newpage
\input{Rozdzialy/oznaczenia}

\pagestyle{fancy}
\newpage
%sets up headers for lefthand and righthand pages. To alter, edit
%these lines and the chaptermark/sectionmark lines above
%\addtolength{\headheight}{3pt} \fancyhead{}
%\fancyhead[LE]{\sl\leftmark} \fancyhead[LO,RE]{\rm\thepage}
%\fancyhead[RO]{\sl\rightmark} \fancyfoot[C,L,E]{}
\pagenumbering{arabic}
\fancyhead[LE,RO]{\slshape \rightmark}
\fancyhead[LO,RE]{\slshape \leftmark}
\fancyfoot[C]{\thepage}
\setlength{\parskip}{1ex} %odstępy między akapitami
%\singlespacing
%\doublespacing
\onehalfspacing
\newpage


%%%%%%%%%%%%%%%%%%%%%%%%%%%%%%%%%%%%%%%%%%%%%%%%%%%%%%%%%%%%%%%%
%%%%%%%%%%%%%%% TUTAJ ZMIEŃ %%%%%%%%%%%%%%%
%%%%%%%%%%%%%% ROZDZIAŁY - zmień nazwy %%%%%%%%%%%%%%%%%%%%%


\addcontentsline{toc}{chapter}{Wstęp}
\chapter{Rozdział 1}

\section{Wprowadzenie}
Autor pracy zdecydował się na stworzenie projektu, który bazuje na zasadach gry \textit{Slay}, ponieważ od zawsze miał zamiar stworzyć własną grę komputerową, a \textit{Slay} jest jedną z gier, która szczególnie przykuła jego uwagę. Jest to tytuł strategiczny, w którym kluczowe znaczenie mają przemyślane decyzje, rozwój zasobów oraz umiejętność przewidywania ruchów przeciwnika. Te elementy strategii i rozgrywki bardzo odpowiadają autorowi, jednak dostrzega on również pewne niedociągnięcia w \textit{Slay}, które sprawiają, że gra nie spełnia wszystkich jego oczekiwań.

Gra \textit{Slay} oferuje tryb wieloosobowy, ale wyłącznie w formie tzw. rozgrywki „hot-seat” – gracze na zmianę wykonują swoje ruchy na tym samym urządzeniu. Dla autora taki tryb wieloosobowy jest niewystarczający, ponieważ brak w nim możliwości rozgrywki w czasie rzeczywistym z innymi graczami online. Autor uważa, że tryb online pozwoliłby na głębszą interakcję między graczami i nadał grze zupełnie nowy wymiar. Jest przekonany, że gra strategiczna tego typu – rozgrywana w czasie rzeczywistym – nie tylko zapewniłaby większe wyzwania, ale też wzbogaciłaby doświadczenie i zaangażowanie graczy.

Wybór pracy opartej na grze \textit{Slay} stanowi także odpowiedź na chęć zdobycia wiedzy i doświadczenia w zakresie tworzenia gier, szczególnie w środowisku, które umożliwia implementację trybu wieloosobowego online. Autor chce połączyć swoją pasję do strategii z umiejętnościami programistycznymi, jednocześnie zaspokajając swoje ambicje, by stworzyć grę bardziej rozbudowaną i angażującą niż pierwowzór. Realizacja tego projektu to także doskonała okazja, by zagłębić się w tematykę programowania sieciowego, synchronizacji danych oraz projektowania przyjaznych dla gracza interfejsów.

Wprowadzenie trybu online wymaga rozważenia różnych aspektów interakcji między graczami, takich jak balans rozgrywki, przeciwdziałanie oszustwom, a także zapewnienie stabilności i płynności działania aplikacji – te wyzwania są dla autora fascynującym polem do rozwoju.

\newpage
\section{Cel pracy}

Ostatecznym celem autora jest stworzenie gry, która zaoferuje innym graczom nie tylko znane już mechanizmy, które są charakterystyczne dla \textit{Slay}, ale także nowe doświadczenia i emocje płynące z możliwości rozgrywki w trybie wieloosobowym online. Praca nad tym projektem jest więc dla niego zarówno realizacją wieloletniego marzenia, jak i cenną lekcją w zakresie programowania oraz projektowania gier.

Aby zrealizować ten cel, autor stawia przed sobą szereg mniejszych, szczegółowych zadań. Jednym z nich jest implementacja systemu generowania map, który umożliwi tworzenie różnorodnych i zbalansowanych plansz do gry. Kluczowym elementem jest także opracowanie systemu ekonomii, który nada głębię mechanikom gry, pozwalając graczom na strategiczne zarządzanie zasobami. Ponadto przewidziane jest wdrożenie mechanizmów zarządzania jednostkami, które będą odzwierciedlać zasady gry \textit{Slay}, ale również uwzględniać nowe możliwości taktyczne. Istotnym wyzwaniem będzie zaprojektowanie stabilnego trybu wieloosobowego online, zapewniającego płynną interakcję między graczami.

W procesie tworzenia gry zostaną również zaimplementowane elementy ułatwiające nawigację, takie jak intuicyjny interfejs użytkownika oraz algorytmy wspomagające balans rozgrywki. Wszystkie te cele mają na celu stworzenie gry, która będzie nie tylko funkcjonalna, ale także angażująca i satysfakcjonująca dla graczy.

\newpage
\section{Teza Pracy}

Dynamiczny rozwój technologii mobilnych oraz rosnące zapotrzebowanie na innowacyjne formy rozrywki sprawiają, że gry komputerowe oparte na mechanikach strategicznych zyskują coraz większą popularność. Niniejsza praca ma na celu wykazanie, że możliwe jest efektywne zaprojektowanie i zaimplementowanie gry komputerowej, inspirowanej mechanikami rozgrywki znanymi z gry \textit{Slay}, przy jednoczesnym zachowaniu jej strategicznego charakteru i intuicyjności. W ramach realizacji tego projektu zostaną przeanalizowane kluczowe aspekty tworzenia gry, takie jak projektowanie interfejsu użytkownika, optymalizacja mechanizmów rozgrywki oraz implementacja funkcji wspierających rozgrywkę wieloosobową online. Wynikiem pracy będzie gra, która nie tylko wiernie odwzoruje istotne cechy „Slay”, ale również wprowadzi elementy innowacyjne, zwiększając jej atrakcyjność w cyfrowym środowisku.

\chapter{Rozdział 2}
\newpage
\section{Tworzenie shaderów}

\subsection{Wprowadzenie do shaderów}

Współczesne silniki renderujące oraz biblioteki graficzne, takie jak \textit{WebGL}, \textit{OpenGL} czy \textit{Vulkan}, opierają się na koncepcji programowalnego procesu renderowania.
Tradycyjnie jego zadaniem było przetwarzanie wierzchołków modeli 3D oraz rasteryzacja geometrii na piksele.
Dzięki zastosowaniu tzw. \textit{shaderów} --- niewielkich programów wykonywanych bezpośrednio na karcie graficznej --- możliwe stało się pełne kontrolowanie sposobu, w jaki dane obiekty są rysowane i oświetlane.

Shader jest programem, który przetwarza dane wejściowe przekazywane z aplikacji (zazwyczaj wierzchołki, kolory, tekstury lub parametry transformacji) i generuje dane wyjściowe odpowiadające za wygląd obrazu widzianego na ekranie.
W praktyce pozwala to na realizację dowolnych efektów wizualnych --- od prostych zmian koloru, po zaawansowane efekty świetlne, przezroczystości czy symulacje fizyczne w grafice.

W ramach projektu \textit{Multiyoy} zastosowano proces renderowania oparty na dwóch typach shaderów:
\begin{itemize}
    \item \textbf{Vertex shader} --- odpowiedzialny za przetwarzanie współrzędnych wierzchołków geometrycznych.
    W projekcie został wykorzystany do generowania pozycji sześciokątów (heksagonów) na mapie oraz obliczeń transformacji przestrzennych.
    \item \textbf{Fragment shader} --- przetwarza każdy piksel (fragment) obrazu.
    W projekcie jego zadaniem jest obliczanie koloru poszczególnych obszarów heksagonu, w tym krawędzi i wewnętrznej części pola.
\end{itemize}

Projekt wykorzystuje najnowszy standard \textbf{WebGL 2.0}, który wspiera język \textbf{GLSL ES 3.00}, umożliwiający pisanie shaderów o złożonej logice i obsługujących typy całkowite oraz tablice struktur.
Takie podejście pozwoliło na stworzenie w pełni programowalnego systemu renderującego, który nie wymaga ręcznego definiowania wszystkich wierzchołków w kodzie aplikacji --- geometrię heksagonu generuje bezpośrednio shader, a wiele jego instancji jest renderowanych w jednym wywołaniu funkcji przy użyciu mechanizmu \textit{instancjonowania}.

Implementacja shaderów w projekcie nie tylko umożliwiła wydajne renderowanie dużych siatek heksagonów,
ale również zapewniła elastyczność podczas tworzenia efektów graficznych, takich jak manipulacja jasnością, nasyceniem czy kolorem krawędzi.
Zastosowanie tej technologii stanowi podstawę całego systemu wizualnego gry i było kluczowym elementem realizacji założonych celów projektowych.

\subsection{Architektura systemu shaderów}

Implementacja shaderów w projekcie \textit{Multiyoy} została zrealizowana z wykorzystaniem technologii \textbf{WebGL 2.0},
umożliwiającej bezpośredni dostęp do sprzętowego procesu renderowania w przeglądarce.
Architektura systemu shaderów została zaprojektowana w sposób modularny, co pozwala na łatwe modyfikowanie i rozbudowę efektów graficznych bez ingerencji w główną logikę gry.

Aplikacja inicjuje kontekst graficzny poprzez obiekt \texttt{WebGL2RenderingContext}, który jest tworzony na elemencie \texttt{canvas}.
W momencie uruchomienia programu ładowane są dwa zewnętrzne pliki: \texttt{vertexShader.glsl} oraz \texttt{fragmentShader.glsl}.
Są one importowane w postaci surowych ciągów tekstowych (\texttt{?raw}) i przekazywane do dedykowanej funkcji kompilującej shadery:

\begin{lstlisting}[language=JavaScript, caption={Kompilacja i linkowanie shaderów w WebGL}]
const vertShader = compileShader(gl.VERTEX_SHADER, vertexShaderSource);
const fragShader = compileShader(gl.FRAGMENT_SHADER, fragmentShaderSource);
gl.attachShader(program, vertShader);
gl.attachShader(program, fragShader);
gl.linkProgram(program);
\end{lstlisting}

Każdy shader jest kompilowany niezależnie, a następnie łączony w jeden program graficzny przy pomocy \texttt{gl.linkProgram(program)}.
W przypadku błędu kompilacji lub linkowania program zatrzymuje działanie i wypisuje odpowiednią informację diagnostyczną.
Takie podejście zapewnia pełną kontrolę nad procesem budowania programu GPU oraz pozwala łatwo wykrywać błędy syntaktyczne na wczesnym etapie.

Po pomyślnym połączeniu shaderów tworzony jest obiekt \texttt{Vertex Array Object (VAO)},
który wiąże wszystkie atrybuty wierzchołków oraz uniformy z programem graficznym.
W projekcie \textit{Multiyoy} do shaderów przekazywane są między innymi następujące dane:
\begin{itemize}
    \item \textbf{u\_mvp} — macierz transformacji 2D (macierz typu \texttt{mat3}), odpowiedzialna za skalowanie, obrót i przesunięcie heksagonów na ekranie,
    \item \textbf{u\_center} — pozycja środka każdego heksagonu, przekazywana jako atrybut wierzchołka,
    \item \textbf{u\_edgeMask} — maska bitowa opisująca, które krawędzie heksagonu mają zostać narysowane,
    \item \textbf{u\_fillColorMask} — maska definiująca sposób podziału kolorów oraz orientację (wertykalną lub horyzontalną),
    \item \textbf{FILL\_COLORS} oraz \textbf{EDGE\_COLORS} — tablice kolorów zapisane w postaci uniformów, wykorzystywane do wypełnienia i krawędzi.
\end{itemize}

Przekazywanie danych odbywa się przy użyciu buforów tablicowych (\texttt{ARRAY\_BUFFER}) tworzonych po stronie GPU.
Dla każdego heksagonu aplikacja generuje pozycję w układzie współrzędnych typu \textit{Axial},
a następnie umieszcza ją w buforze za pomocą:

\begin{lstlisting}[language=JavaScript, caption={Przekazanie danych do bufora GPU}]
gl.bufferData(gl.ARRAY_BUFFER, centerData, gl.STATIC_DRAW);
\end{lstlisting}

Dzięki temu możliwe jest renderowanie tysięcy instancji w jednym wywołaniu:

\begin{lstlisting}[language=JavaScript, caption={Renderowanie instancjonowane hexagonów}]
gl.drawArraysInstanced(gl.TRIANGLE_FAN, 0, 8, centers.length);
\end{lstlisting}

Mechanizm \textbf{instancjonowania} pozwala na znaczną redukcję kosztów obliczeniowych,
ponieważ współdzielona geometria heksagonu jest przetwarzana tylko raz,
a każdy kolejny heksagon korzysta z tych samych danych wierzchołków, różniąc się jedynie pozycją i kolorem.
Wszystkie transformacje położenia i skali są wykonywane w \textit{vertex shaderze}, natomiast kolory oraz renderowanie krawędzi są realizowane w \textit{fragment shaderze}.

Cały system shaderów został zaprojektowany jako elastyczny moduł,
który można rozbudowywać o kolejne efekty, takie jak dynamiczne cieniowanie, podświetlenia czy wizualne zaznaczenia pól.
Dzięki oddzieleniu logiki renderowania od logiki gry, architektura zachowuje spójność,
a modyfikacje wyglądu nie wpływają na mechanikę rozgrywki.
W praktyce taka konstrukcja umożliwia szybkie eksperymenty z kolorystyką, grubością krawędzi czy trybami prezentacji mapy,
bez konieczności przebudowy całego systemu graficznego.

\subsection{Etapy rozwoju shaderów}

Prace nad systemem shaderów rozpoczęły się od prostych ćwiczeń mających na celu zrozumienie przepływu danych pomiędzy procesorem a kartą graficzną oraz poznanie zasad działania \textit{WebGL 2.0}.
Pierwsze próby polegały na rysowaniu trójkątów i kwadratów przy użyciu tablic wierzchołków oraz bufora indeksów.
Na tym etapie poznano mechanizmy interpolacji kolorów w shaderze fragmentów i wprowadzono obsługę podstawowych transformacji figur.
Uzyskane doświadczenia pozwoliły zbudować prosty, lecz w pełni funkcjonalny system rysowania kształtów w przeglądarce.

Kolejnym krokiem było generowanie geometrii przybliżonego koła.
Figura została utworzona w trybie \texttt{gl.TRIANGLE\_FAN}, gdzie współrzędne wierzchołków obliczano z użyciem funkcji trygonometrycznych.
W dalszym etapie część obliczeń przeniesiono do \textit{vertex shadera}, który samodzielnie wyznaczał położenie punktów okręgu.
Dzięki temu zredukowano ilość danych przesyłanych pomiędzy CPU a GPU oraz zoptymalizowano proces renderowania.
Dodatkowo wprowadzono macierz transformacji \texttt{u\_mvp}, która umożliwiła skalowanie, obracanie i przesuwanie figury oraz interaktywną obsługę myszy i kółka przewijania.

Po opanowaniu geometrii przystąpiono do rysowania pojedynczego heksagonu — podstawowej jednostki mapy.
Współrzędne jego wierzchołków generowano matematycznie bezpośrednio w shaderze,
a \textit{fragment shader} nadawał jednolity kolor przekazywany jako uniform.
Następnie rozszerzono projekt o możliwość obrysowywania krawędzi poprzez wyznaczanie odległości poszczególnych pikseli od linii brzegowych.
Ten etap, mimo swojej pozornej prostoty, okazał się najbardziej czasochłonny i wymagał dopracowania wielu detali matematycznych.

W dalszej części projektu wprowadzono tzw. maski krawędziowe, pozwalające włączać i wyłączać konkretne krawędzie heksagonu.
Dzięki maskom bitowym udało się wyeliminować podwójne obramowania na stykach pól i znacznie poprawić czytelność mapy.
Kolejnym krokiem było dodanie podziału kolorystycznego na dwie części — pionową lub poziomą — co pozwoliło tworzyć setki unikalnych kombinacji barw dla wielu graczy na rozległych mapach.

Po przygotowaniu warstwy wizualnej skupiono się na wydajności.
Zamiast rysować każdy heksagon osobno, zastosowano instancjonowanie, co pozwoliło renderować całą mapę – setki tysięcy heksagonów – w jednym wywołaniu funkcji \texttt{drawArraysInstanced}.
Dzięki temu wykonywanie obliczeń przeniesiono niemal w całości na GPU, a liczba klatek na sekundę pozostała stabilna nawet przy bardzo dużych mapach.
Układ współrzędnych Axial umożliwił przy tym logiczne rozmieszczenie heksagonów bez potrzeby przechowywania ich pozycji w pamięci w sposób nieuporządkowany.

W późniejszym etapie wprowadzono globalny system kolorów.
Każdy heksagon przestał przechowywać własne wartości RGB i zamiast tego korzystał z indeksów wskazujących pozycję w tablicach \texttt{FILL\_COLORS} i \texttt{EDGE\_COLORS}.
Zastosowanie uniformów dla palet kolorystycznych znacząco obniżyło liczbę danych przesyłanych z procesora i uprościło zarządzanie wyglądem mapy.
Całość była renderowana w przestrzeni barw OKLCH, która zapewniała lepsze odwzorowanie percepcyjne kolorów niż tradycyjny RGB.

Ostatnim krokiem rozwoju shaderów było dodanie interaktywnej manipulacji jasnością i nasyceniem barw.
Do interfejsu użytkownika wprowadzono suwaki, które w czasie rzeczywistym modyfikowały parametry palety, umożliwiając globalną zmianę tonacji mapy.
Dzięki wykorzystaniu GPU obliczenia były wykonywane natychmiastowo,
a użytkownik mógł płynnie dostosowywać wygląd gry bez utraty wydajności.

Finalny efekt to system zdolny do generowania i wyświetlania milionów heksagonów w czasie rzeczywistym, z pełną kontrolą nad wyglądem, kolorystyką oraz możliwością interakcji.

\subsection{Podsumowanie}

Rozwój systemu shaderów dla projektu \textit{Multiyoy} stanowił długotrwały proces,
w trakcie którego prosty zestaw funkcji graficznych został przekształcony w kompletny, wydajny i skalowalny silnik renderujący.
Całość rozpoczęła się od nauki podstaw działania \textit{WebGL} i rysowania prostych figur,
a zakończyła na opracowaniu złożonego systemu potrafiącego generować i wyświetlać miliony heksagonów w czasie rzeczywistym.

Kolejne miesiące pracy przyniosły szereg udoskonaleń – od przenoszenia obliczeń do shaderów,
poprzez dodanie ramek i kolorowych masek, aż po zastosowanie instancjonowania, które wielokrotnie zwiększyło wydajność renderowania.
W późniejszych etapach stworzono globalny system palet kolorów i mechanizm dynamicznej manipulacji jasnością oraz nasyceniem,
co pozwoliło na pełną kontrolę nad wyglądem mapy bez utraty płynności.

Stworzony system jest modularny, elastyczny i łatwy w rozbudowie.
Każdy jego element – od generowania geometrii po przetwarzanie kolorów – może być rozwijany niezależnie,
co stanowi solidny fundament do dalszych prac nad grą.
Projekt pokazał również praktyczną wartość shaderów w nowoczesnych aplikacjach webowych:
pozwalają one osiągnąć poziom wydajności i jakości graficznej zarezerwowany dotąd dla klasycznych silników graficznych.

Efektem końcowym jest stabilny i optymalny moduł graficzny,
który nie tylko spełnia wymagania projektu \textit{Multiyoy},
ale może zostać łatwo wykorzystany lub rozwinięty w kolejnych wersjach gry i przyszłych projektach badawczych.
\chapter{Rozdział 3}
\input{Rozdzialy/rozdzial3}
\chapter{Podsumowanie}
\input{Rozdzialy/rozdzial4}
\addcontentsline{toc}{chapter}{Załączniki}
\chapter*{Załączniki}
\input{Rozdzialy/zalaczniki}
%%%%%%%%%%%%%%%%%%%%%%%%%%%%%%%%%%%%%%%%%%%%%%%%%%%%%%%%%%%%%%%%%




%   this is for BibTeX.  remove if you plan to write the references in the document
\bibliographystyle{plplain}
\bibliography{bibliografia}

\addcontentsline{toc}{chapter}
         {\protect\numberline{Bibliografia\hspace{-96pt}}}
%streszczenie
    \addcontentsline{toc}{chapter}{Streszczenie}
    \chapter*{}

\textbf{STRESZCZENIE}

Imię Nazwisko

\begin{center}
    TEMAT PRACY
\end{center}

Tekst streszczenia w języku polskim/obcym.

Tekst .................... 12 Times New Roman

\hfill


\textbf{SUMMARY}

Name Surname

\begin{center}
    TOPIC
\end{center}


Tekst streszczenia w języku polskim/obcym.

Tekst .................... 12 Times New Roman

\end{document}